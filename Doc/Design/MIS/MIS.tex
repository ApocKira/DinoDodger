
\documentclass[12pt, titlepage]{article}

\usepackage{fullpage}
\usepackage[round]{natbib}
\usepackage{multirow}
\usepackage{booktabs}
\usepackage{tabularx}
\usepackage{graphicx}
\usepackage{float}
\usepackage{hyperref}
\hypersetup{
    colorlinks,
    citecolor=black,
    filecolor=black,
    linkcolor=red,
    urlcolor=blue
}
\usepackage[round]{natbib}

\newcounter{acnum}
\newcommand{\actheacnum}{AC\theacnum}
\newcommand{\acref}[1]{AC\ref{#1}}

\newcounter{ucnum}
\newcommand{\uctheucnum}{UC\theucnum}
\newcommand{\uref}[1]{UC\ref{#1}}

\newcounter{mnum}
\newcommand{\mthemnum}{M\themnum}
\newcommand{\mref}[1]{M\ref{#1}}

\title{SE 3XA3: MIS \\DinoDodger}

\author{Team 39, S.R.A Squad
		\\ Shrey Mittal, mittas1
		\\ Razan Abujarad, abujarar
		\\ Allen Yang, yangz18
}

\date{9 November, 2017}

\begin{document}

\maketitle

\pagenumbering{roman}
\tableofcontents
\listoftables
\listoffigures

\begin{table}[bp]
\caption{\bf Revision History}
\begin{tabularx}{\textwidth}{p{3cm}p{2cm}X}
\toprule {\bf Date} & {\bf Version} & {\bf Notes}\\
\midrule
\date{2017/11/9} & 1.0 & Creating MIS\\
\date{2017/12/04} &  2.0 & MIS Revision \\
\bottomrule
\end{tabularx}
\end{table}

\newpage
\section {UI Module}
\subsection{Module}
User Interface
\subsection {Uses}
N/A
\subsection {Syntax}
\subsubsection {Exported Access Programs}
\begin{tabular}{| l | l | l | l |}
\hline
\textbf{Routine name} & \textbf{In} & \textbf{Out} & \textbf{Exceptions}\\
\hline
start & Stage & Scene & none\\
\hline
playButtonSelected & Event & none & none\\
\hline
char1Selected & Event & none & none\\
\hline
char2Selected & Event & none & none\\
\hline
char3Selected & Event & none & none\\
\hline
landscape1Selected & Event & none & none\\
\hline
landscape2Selected & Event & none & none\\
\hline
landscape3Selected & Event & none & none\\
\hline
main & none & none & none\\
\hline
\end{tabular}
\subsection {Semantics}
\subsubsection {State Variables}
button1 := Button\\
button2 := Button\\
button3 := Button\\
button4 := Button\\
button5 := Button\\
button6 := Button\\
button7 := Button\\
char1 := String\\
char2 := String\\
char3 := String\\
char := String\\
landscape := String\\
landscape1 := String\\
landscape2 := String\\
landscape3 := String\\
scene1 := Scene\\
scene2 := Scene\\
scene3 := Scene\\
\subsubsection {State Invariant}
none
\subsubsection {Access Routine Semantics}
start(Stage):
\begin{itemize}
\item transition: Creation of stage(window) with scene
\item exception: none
\end{itemize}
char1Selected:
\begin{itemize}
\item transition: char = char1
\item exception: none
\end{itemize}
char1Selected:
\begin{itemize}
\item transition: char = char2
\item exception: none
\end{itemize}
char3Selected:
\begin{itemize}
\item transition: char = char3
\item exception: none
\end{itemize}
landscape1Selected:
\begin{itemize}
\item transition: landscape = landscape1
\item exception: none
\end{itemize}
landscape2Selected:
\begin{itemize}
\item transition: landscape = landscape2
\item exception: none
\end{itemize}
landscape3Selected:
\begin{itemize}
\item transition: landscape = landscape3
\item exception: none
\end{itemize}
okayButtonSelected:
\begin{itemize}
\item transition: Goes from scene1 to scene2
\item exception: none
\end{itemize}



\newpage
\section {Sprite Animation Module}
\subsection{Template Module}
Sprite Animation
\subsection {Uses}
N/A
\subsection {Syntax}
\subsubsection {Exported Constants}
imageView: ImageView\\
COUNT: int\\
COLUMNS: int\\
OFFSET\_X: int\\
OFFSET\_Y: int\\
WIDTH: int\\
HEIGHT: int\\
\subsubsection {Exported Types}
SpriteAnimation 
\subsubsection {Exported Access Programs}
\begin{tabular}{| l | l | l | l |}
\hline
\textbf{Routine name} & \textbf{In} & \textbf{Out} & \textbf{Exceptions}\\
\hline
SpriteAnimation & ImageView, Duration, int, int, int, int, int, int & SpriteAnimation & none\\
\hline
setOffSetX & int & none & none\\
\hline
setOffSetY & int & none & none\\
\hline
interpolate & double & none & none\\
\hline
\end{tabular}
\subsection {Semantics}
\subsubsection {State Variables}
imageView: ImageView\\
duration: Duration\\
count: int\\
columns: int\\
offSetX: int\\
offSetY: int\\
width: inti\\
height: int\\

\subsubsection {State Invariant}
none
\subsubsection {Assumptions}
none
\subsubsection {Access Routine Semantics}
SpriteAnimation(imageView, duration, count, columns, offSetX, offSetY, width, height):
\begin{itemize}
\item transition: imageView, setCycleDuration(duration), COUNT, COLUMNS, OFFSET\_X, OFFSET\_Y, WIDTH, HEIGHT  := imageView, duration, count, columns, offSetX, offSetY, width, height
\item output: $out := \mathit{self}$
\item exception: none
\end{itemize}
\noindent setOffSetX($x$):
\begin{itemize}
\item transition: OFFSET\_X := x
\item output: none
\item exception: none
\end{itemize}
\noindent setOffSetY($y$):
\begin{itemize}
\item transition: OFFSET\_Y := y
\item output: none
\item exception: none
\end{itemize}
\noindent interpolate($frac$):
\begin{itemize}
\item transition: imageView is set to new viewport using a Rectangle2D object wifth values x, y, width, height, where x and y are as follows: \\ \\
index := min(floor(COUNT*frac, COUNT-1)) \\ 
x := (index mod(COLUMNS))*WIDTH+OFFSET\_X \\
y := (index/COLUMNS)*HEIGHT+OFFSET\_Y \\
\item output: none
\item exception: none
\end{itemize}

\newpage
\section {Character Module}
\subsection{Template Module}
PointT
\subsection {Uses}
SpriteAnimation
\subsection {Syntax}
\subsubsection{Exported Constants}
imageView: ImageView\\
COUNT: int\\
COLUMNS: int\\
OFFSET\_X: int\\
OFFSET\_Y: int\\
WIDTH: int\\
HEIGHT: int\\
\subsubsection {Exported Types}
Character
\subsubsection {Exported Access Programs}
\begin{tabular}{| l | l | l | l |}
\hline
\textbf{Routine name} & \textbf{In} & \textbf{Out} & \textbf{Exceptions}\\
\hline
Character & ImageView & Character & none\\
\hline
jump & none & Character Animated to Jump & none\\
\hline
\end{tabular}
\subsection {Semantics}
\subsubsection {State Variables}
$animation$ := SpriteAnimation
\subsubsection {State Invariant}
none
\subsubsection {Assumptions}
none
\subsubsection {Access Routine Semantics}
Character($imageView$):
\begin{itemize}
\item transition: $imageView := imageView$ 
\item output: Outputs Animation onto an imageView
\item exception: none
\end{itemize}
\noindent jump:
\begin{itemize}
\item transition: none
\item output: Character jumps
\item exception: none
\end{itemize}

\newpage
\section {Animation Module}
\subsection{Template Module}
Animation
\subsection {Uses}
Character, DinoDodger
\subsection {Syntax}
\subsubsection {Exported Types}
Animation
\subsubsection {Exported Access Programs}
\begin{tabular}{| l | l | l | l |}
\hline
\textbf{Routine name} & \textbf{In} & \textbf{Out} & \textbf{Exceptions}\\
\hline
start & Stage & Window & none\\
\hline
main & none & all arguments launched & none\\
\hline
\end{tabular}
\subsection {Semantics}
\subsubsection {State Variables}
$goAnimation$ := SpriteAnimation\\
$goUp$ := SpriteAnimation\\
$imageView1$ := ImageView\\
$imageView2$ := ImageView\\
$imageView3$ := ImageView\\
$animation$ := SpriteAnimation\\
\subsubsection {State Invariant}
While character has not intersected obstacle, continue.
\subsubsection {Assumptions}
none
\subsubsection {Access Routine Semantics}
start($imageView$):
\begin{itemize}
\item transition: $imageView := imageView$ 
\item output: Outputs Character Animation, Obstacles and Background onto Scene
\item exception: none
\end{itemize}
\noindent main:
\begin{itemize}
\item transition: All arguments launched
\item output: Gameplay mode
\item exception: none
\end{itemize}

\newpage
\section {DinoDodger Module}
\subsection{Template Module}
DinoDodger
\subsection {Uses}
User Interface Module
\subsection {Syntax}
\subsubsection {Exported Types}
none; this module is medium of communication
\subsubsection {Exported Access Programs}
\begin{tabular}{| l | l | l | l |}
\hline
\textbf{Routine name} & \textbf{In} & \textbf{Out} & \textbf{Exceptions}\\
\hline
getCharacterSelected & String & Image & none\\
\hline
getLandScapeSelected & String &Image & none\\
\hline
getScore& none & int & none\\
\hline
getHighScore & none & int & none\\
\hline
\end{tabular}
\subsection {Semantics}
\subsubsection {State Variables}
$POINTS$ := int\\
$HIGHSCORE$ := int\\
$landScape\_1$ := Image\\
$landScape\_2$ := Image\\
$landScape\_3$ := Image\\
$character\_1$ := Image\\
$character\_2$ := Image\\
$character\_3$ := Image\\
\subsubsection{State Invariant}
none
\subsubsection{Assumptions}
none
\subsubsection{Access Routine Semantics}
getCharacterSelected(character)
\begin{itemize}
\item transition: returns Image based on String equivalence
\item output:  character\_1, character\_2, character\_3  based on equivalence of character
\item exception: none
\end{itemize}
\noindent getLandScapeSelected(landScape)
\begin{itemize}
\item transition: returns Image based on String equivalence
\item output: landScape\_1, landScape\_2, landScape\_3  based on equivalence of character
\item exception: none
\end{itemize}
\noindent getScore:
\begin{itemize}
\item transition: HIGHSCORE = POINTS
\item output: POINTS
\item exception: POINTS < HIGHSCORE in which case transition does not occur.
\end{itemize}
\noindent getHighScore:
\begin{itemize}
\item transition: none
\item output: HIGHSCORE
\item exception: none
\end{itemize}

\newpage
\section {UI Module}
\subsection{Module}
User Interface
\subsection {Uses}
N/A
\subsection {Syntax}
\subsubsection {Exported Access Programs}
\begin{tabular}{| l | l | l | l |}
\hline
\textbf{Routine name} & \textbf{In} & \textbf{Out} & \textbf{Exceptions}\\
\hline
start & Stage & Scene & none\\
\hline
playButtonSelected & Event & none & none\\
\hline
char1Selected & Event & none & none\\
\hline
char2Selected & Event & none & none\\
\hline
char3Selected & Event & none & none\\
\hline
landscape1Selected & Event & none & none\\
\hline
landscape2Selected & Event & none & none\\
\hline
landscape3Selected & Event & none & none\\
\hline
main & none & none & none\\
\hline
\end{tabular}
\subsection {Semantics}
\subsubsection {State Variables}
button1 := Button\\
button2 := Button\\
button3 := Button\\
button4 := Button\\
button5 := Button\\
button6 := Button\\
button7 := Button\\
char1 := String\\
char2 := String\\
char3 := String\\
char := String\\
landscape := String\\
landscape1 := String\\
landscape2 := String\\
landscape3 := String\\
scene1 := Scene\\
scene2 := Scene\\
scene3 := Scene\\
\subsubsection {State Invariant}
none
\subsubsection {Access Routine Semantics}
start(Stage):
\begin{itemize}
\item transition: Creation of stage(window) with scene
\item exception: none
\end{itemize}
char1Selected:
\begin{itemize}
\item transition: char = char1
\item exception: none
\end{itemize}
char1Selected:
\begin{itemize}
\item transition: char = char2
\item exception: none
\end{itemize}
char3Selected:
\begin{itemize}
\item transition: char = char3
\item exception: none
\end{itemize}
landscape1Selected:
\begin{itemize}
\item transition: landscape = landscape1
\item exception: none
\end{itemize}
landscape2Selected:
\begin{itemize}
\item transition: landscape = landscape2
\item exception: none
\end{itemize}
landscape3Selected:
\begin{itemize}
\item transition: landscape = landscape3
\item exception: none
\end{itemize}
okayButtonSelected:
\begin{itemize}
\item transition: Goes from scene1 to scene2
\item exception: none
\end{itemize}



\newpage
\section {Sprite Animation Module}
\subsection{Template Module}
Sprite Animation
\subsection {Uses}
N/A
\subsection {Syntax}
\subsubsection {Exported Constants}
imageView: ImageView\\
COUNT: int\\
COLUMNS: int\\
OFFSET\_X: int\\
OFFSET\_Y: int\\
WIDTH: int\\
HEIGHT: int\\
\subsubsection {Exported Types}
SpriteAnimation 
\subsubsection {Exported Access Programs}
\begin{tabular}{| l | l | l | l |}
\hline
\textbf{Routine name} & \textbf{In} & \textbf{Out} & \textbf{Exceptions}\\
\hline
SpriteAnimation & ImageView, Duration, int, int, int, int, int, int & SpriteAnimation & none\\
\hline
setOffSetX & int & none & none\\
\hline
setOffSetY & int & none & none\\
\hline
interpolate & double & none & none\\
\hline
\end{tabular}
\subsection {Semantics}
\subsubsection {State Variables}
imageView: ImageView\\
duration: Duration\\
count: int\\
columns: int\\
offSetX: int\\
offSetY: int\\
width: inti\\
height: int\\

\subsubsection {State Invariant}
none
\subsubsection {Assumptions}
none
\subsubsection {Access Routine Semantics}
SpriteAnimation(imageView, duration, count, columns, offSetX, offSetY, width, height):
\begin{itemize}
\item transition: imageView, setCycleDuration(duration), COUNT, COLUMNS, OFFSET\_X, OFFSET\_Y, WIDTH, HEIGHT  := imageView, duration, count, columns, offSetX, offSetY, width, height
\item output: $out := \mathit{self}$
\item exception: none
\end{itemize}
\noindent setOffSetX($x$):
\begin{itemize}
\item transition: OFFSET\_X := x
\item output: none
\item exception: none
\end{itemize}
\noindent setOffSetY($y$):
\begin{itemize}
\item transition: OFFSET\_Y := y
\item output: none
\item exception: none
\end{itemize}
\noindent interpolate($frac$):
\begin{itemize}
\item transition: imageView is set to new viewport using a Rectangle2D object wifth values x, y, width, height, where x and y are as follows: \\ \\
index := min(floor(COUNT*frac, COUNT-1)) \\ 
x := (index mod(COLUMNS))*WIDTH+OFFSET\_X \\
y := (index/COLUMNS)*HEIGHT+OFFSET\_Y \\
\item output: none
\item exception: none
\end{itemize}

\newpage
\section {Character Module}
\subsection{Template Module}
PointT
\subsection {Uses}
SpriteAnimation
\subsection {Syntax}
\subsubsection{Exported Constants}
imageView: ImageView\\
COUNT: int\\
COLUMNS: int\\
OFFSET\_X: int\\
OFFSET\_Y: int\\
WIDTH: int\\
HEIGHT: int\\
\subsubsection {Exported Types}
Character
\subsubsection {Exported Access Programs}
\begin{tabular}{| l | l | l | l |}
\hline
\textbf{Routine name} & \textbf{In} & \textbf{Out} & \textbf{Exceptions}\\
\hline
Character & ImageView & Character & none\\
\hline
jump & none & Character Animated to Jump & none\\
\hline
\end{tabular}
\subsection {Semantics}
\subsubsection {State Variables}
$animation$ := SpriteAnimation
\subsubsection {State Invariant}
none
\subsubsection {Assumptions}
none
\subsubsection {Access Routine Semantics}
Character($imageView$):
\begin{itemize}
\item transition: $imageView := imageView$ 
\item output: Outputs Animation onto an imageView
\item exception: none
\end{itemize}
\noindent jump:
\begin{itemize}
\item transition: none
\item output: Character jumps
\item exception: none
\end{itemize}

\newpage
\section {PlayScene Module}
\subsection{Template Module}
PlayScene
\subsection {Uses}
Character, DinoDodger
\subsection {Syntax}
\subsubsection {Exported Types}
PlayScene
\subsubsection {Exported Access Programs}
\begin{tabular}{| l | l | l | l |}
\hline
\textbf{Routine name} & \textbf{In} & \textbf{Out} & \textbf{Exceptions}\\
\hline
start & Stage & Window & none\\
\hline
main & none & all arguments launched & none\\
\hline
\end{tabular}
\subsection {Semantics}
\subsubsection {State Variables}
$goAnimation$ := SpriteAnimation\\
$goUp$ := SpriteAnimation\\
$imageView1$ := ImageView\\
$imageView2$ := ImageView\\
$imageView3$ := ImageView\\
$animation$ := SpriteAnimation\\
\subsubsection {State Invariant}
While character has not intersected obstacle, continue.
\subsubsection {Assumptions}
none
\subsubsection {Access Routine Semantics}
start($imageView$):
\begin{itemize}
\item transition: $imageView := imageView$ 
\item output: Outputs Character Animation, Obstacles and Background onto Scene
\item exception: none
\end{itemize}
\noindent main:
\begin{itemize}
\item transition: All arguments launched
\item output: Gameplay mode
\item exception: none
\end{itemize}

\newpage
\section {DinoDodger Module}
\subsection{Template Module}
DinoDodger
\subsection {Uses}
User Interface Module
\subsection {Syntax}
\subsubsection {Exported Types}
none; this module is medium of communication
\subsubsection {Exported Access Programs}
\begin{tabular}{| l | l | l | l |}
\hline
\textbf{Routine name} & \textbf{In} & \textbf{Out} & \textbf{Exceptions}\\
\hline
getCharacterSelected & String & Image & none\\
\hline
getLandScapeSelected & String &Image & none\\
\hline
getScore& none & int & none\\
\hline
getHighScore & none & int & none\\
\hline
\end{tabular}
\subsection {Semantics}
\subsubsection {State Variables}
$POINTS$ := int\\
$HIGHSCORE$ := int\\
$landScape\_1$ := Image\\
$landScape\_2$ := Image\\
$landScape\_3$ := Image\\
$character\_1$ := Image\\
$character\_2$ := Image\\
$character\_3$ := Image\\
\subsubsection{State Invariant}
none
\subsubsection{Assumptions}
none
\subsubsection{Access Routine Semantics}
getCharacterSelected(character)
\begin{itemize}
\item transition: returns Image based on String equivalence
\item output:  character\_1, character\_2, character\_3  based on equivalence of character
\item exception: none
\end{itemize}
\noindent getLandScapeSelected(landScape)
\begin{itemize}
\item transition: returns Image based on String equivalence
\item output: landScape\_1, landScape\_2, landScape\_3  based on equivalence of character
\item exception: none
\end{itemize}
\noindent getScore:
\begin{itemize}
\item transition: HIGHSCORE = POINTS
\item output: POINTS
\item exception: POINTS < HIGHSCORE in which case transition does not occur.
\end{itemize}
\noindent getHighScore:
\begin{itemize}
\item transition: none
\item output: HIGHSCORE
\item exception: none
\end{itemize}

\newpage

\textcolor{red}{
\section {Cactus Module}
\subsection{Template Module}
N/A
\subsection {Uses}
UI
\subsection {Syntax}
\subsubsection{Exported Constants}
cactus1 : Image \\
cactus2 : Image \\
cactus3 : Image \\
cactus4 : Image \\
\subsubsection {Exported Types}
Cactus
\subsubsection {Exported Access Programs}
\begin{tabular}{| l | l | l | l |}
\hline
\textbf{Routine name} & \textbf{In} & \textbf{Out} & \textbf{Exceptions}\\
\hline
Cactus & none & none & none\\
\hline
getRandomCactus & none & Cactus ImageView & none\\
\hline
\end{tabular}
\subsection {Semantics}
\subsubsection {State Variables}
N/A
\subsubsection {State Invariant}
none
\subsubsection {Assumptions}
none
\subsubsection {Access Routine Semantics}
Cactus
\begin{itemize}
\item transition: $image := image$ 
\item output: Outputs cactus onto an image
\item exception: none
\end{itemize}
\noindent getRandomCactus:
\begin{itemize}
\item transition: none
\item output: Cactus object
\item exception: none }
\end{itemize}


\newpage

\textcolor{red}{
\section {Pteradactyl Module}
\subsection{Template Module}
N/A
\subsection {Uses}
SpriteAnimation
\subsection {Syntax}
\subsubsection{Exported Constants}
imageView: ImageView\\
COUNT: int\\
COLUMNS: int\\
OFFSET\_X: int\\
OFFSET\_Y: int\\
WIDTH: int\\
HEIGHT: int\\
\subsubsection {Exported Types}
Pteradactyl
\subsubsection {Exported Access Programs}
\begin{tabular}{| l | l | l | l |}
\hline
\textbf{Routine name} & \textbf{In} & \textbf{Out} & \textbf{Exceptions}\\
\hline
Pteradactyl & ImageView & Pteradactyl & none\\
\hline
getRandomHeight & none & Random int & none\\
\hline
\end{tabular}
\subsection {Semantics}
\subsubsection {State Variables}
$animation$ := SpriteAnimation
\subsubsection {State Invariant}
none
\subsubsection {Assumptions}
none
\subsubsection {Access Routine Semantics}
Pteradactyl($imageView$):
\begin{itemize}
\item transition: $imageView := imageView$ 
\item output: Outputs Pteradactyl Animation onto an imageView
\item exception: none
\end{itemize}
\noindent getRandomHeight:
\begin{itemize}
\item transition: none
\item output: Random integer
\item exception: none }
\end{itemize}

\newpage

\textcolor{red}{
\section {PointCounter Module}
\subsection{Template Module}
N/A 
\subsection {Uses}
N/A
\subsection {Syntax}
\subsubsection{Exported Constants}
points : int\\
highscore : int\\
timer : Timer \\
sc : Scanner 
\subsubsection {Exported Types}
PointCounter
\subsubsection {Exported Access Programs}
\begin{tabular}{| l | l | l | l |}
\hline
\textbf{Routine name} & \textbf{In} & \textbf{Out} & \textbf{Exceptions}\\
\hline
PointCounter & int, int & PointCounter & none\\
\hline
hasCollided & none & none & none\\
\hline
showPoints & none & int & none\\
\hline
showHighScore & none & int & none\\
\hline
updateHighScore & none & none & none\\
\hline
reset & none & none & none\\
\hline
run & none & none & none\\
\hline
\end{tabular}
\subsection {Semantics}
\subsubsection {State Variables}
none
\subsubsection {State Invariant}
none
\subsubsection {Assumptions}
none
\subsubsection {Access Routine Semantics}
PointCounter(int, int):
\begin{itemize}
\item transition: $points, highscore := int, int$ 
\item output: PointCounter
\item exception: none
\end{itemize}
\noindent hasCollided:
\begin{itemize}
\item transition: none
\item output: counter stopped
\item exception: none
\end{itemize}

\noindent showPoints:
\begin{itemize}
\item transition: none
\item output: points
\item exception: none
\end{itemize}

\noindent showHighScore:
\begin{itemize}
\item transition: none
\item output: highscore
\item exception: none
\end{itemize}

\noindent updateHighScore:
\begin{itemize}
\item transition: $highscore := points$
\item output: none
\item exception: none
\end{itemize}

\noindent reset:
\begin{itemize}
\item transition: $points, highscore := 0, 0$
\item output: none
\item exception: none
\end{itemize}

\noindent run:
\begin{itemize}
\item transition: $points := points + 1$
\item output: none
\item exception: none }
\end{itemize} 


\end{document}